\usepackage{libertine}
\usepackage{libertinust1math}
\renewcommand*\ttdefault{txtt}
\newcommand{\vnormord}[1]{\mathpunct{}\,\mathopen\vdots\,#1\,\mathclose\vdots\,\mathpunct{}}

\usepackage{bm}
\usepackage{isomath}
\usepackage{siunitx}
\usepackage{subdepth}
\usepackage{tikz}
\usetikzlibrary{calc, shapes}
\usepackage[T1]{fontenc}

\renewcommand*\oldstylenums[1]{\textosf{#1}}
\AtBeginDocument{\renewbibmacro{in:}{}}

\providecommand{\otherDelta}{\Delta}
\providecommand{\otherpi}{\pi}
\providecommand{\upDelta}{\otherDelta}
\providecommand{\uppi}{\otherpi}
\newcommand{\D}{\operatorname{d\!}\mathord{}}
\newcommand{\E}{\mathrm{e}}
\newcommand{\I}{\mathrm{i}}
\newcommand{\PI}{\uppi}
\newcommand{\bigo}{\mathcal{O}}
\newcommand{\symm}{\mathcal{S}}
\newcommand{\antisymm}{\mathcal{A}}
\newcommand{\comm}[1]{\left[ #1 \right]}
\newcommand{\anticomm}[1]{\left\{ #1 \right\}}
\providecommand{\normord}[1]{\mathpunct{}\mathopen: #1 \mathclose:\mathpunct{}}
\providecommand{\vnormord}[1]{\mathpunct{}\mathopen\vdots #1 \mathclose\vdots\mathpunct{}}
\newcommand{\bra}[1]{\langle #1 \mathclose|}
\newcommand{\ket}[1]{\mathopen| #1 \rangle}
\newcommand{\bkt}[1]{\langle #1 \rangle}
\newcommand{\jweight}[1]{\breve{\ifx#1j\jmath\else#1\fi}}
\newcommand{\tridelta}[3]{\begin{Bmatrix} #1 & #2 & #3 \end{Bmatrix}}

\let\originalleft\left
\let\originalright\right
\def\left#1{\mathopen{}\originalleft#1}
\def\right#1{\originalright#1\mathclose{}}

\def\thedegreeprogram{Physics---Doctor of Philosophy}

% ----------------------------------------------------------------------------
%
% TikZ macros
% ===========

% The \lcontr and \rcontr commands are used to display Wick contractions (the
% left and right parts, respectively).  The syntax is given by:
%
%     \lcontr{INDEX}{OPERATOR}
%     \rcontr[GAP]{INDEX}{OPERATOR}
%
% where:
%
% * INDEX is a unique number used to both identify and specify the height of
%   the contraction.  The number only needs to be unique within the same
%   product.  For best display, use natural numbers starting from one.
%
% * OPERATOR is the operator, e.g. \hat{a}.
%
% * GAP is the small gap between the operator and the end of the line.
%   Default value is 0.03.
%
% Note that this will require two TeX passes to be displayed properly.
%
\providecommand{\lcontr}[2]{
  \tikz[baseline = (contraction-#1-left.base), remember picture]
  \node[inner sep = 0] (contraction-#1-left) {\ensuremath{#2}};
}
\providecommand{\rcontr}[3][.03]{
  \tikz[baseline = (contraction-#2-right.base), remember picture]
  \node[inner sep = 0] (contraction-#2-right) {\ensuremath{#3}};
  \pgfmathsetmacro{\contractiondirection}{#2 > 0 ? "north" : "south"}
  \pgfmathsetmacro{\contractiongap}{.03}
  \pgfmathsetmacro{\contractionoffset}{
    #2 > 0 ? \contractiongap : -\contractiongap - .1
  }
  \pgfmathsetmacro{\contractionheight}{
    #2 > 0 ? #2 * .08 + .45 : #2 * .08 - .22
  }
  \begin{tikzpicture}[overlay, remember picture, thick]
    \draw let \p1 = (contraction-#2-left.\contractiondirection) in
             ($ (contraction-#2-left.\contractiondirection)
                + (0, \contractionoffset) $)
          -- ($ (\x1, \contractionheight) $)
          -| ($ (contraction-#2-right.\contractiondirection)
                + (0, \contractionoffset) $);
  \end{tikzpicture}
}
